% main_configuration.tex
\documentclass{article}

% ========
% 	packages		
% ========
\usepackage[utf8]{inputenc} % to use special characters
\usepackage{geometry} % page size
\usepackage{fancyhdr} % page style
\usepackage{float} % force image location
\usepackage{graphicx, wrapfig, caption, subcaption} % figures
\usepackage{amsmath, mathtools, esvect} % math 
\usepackage{amssymb, amsfonts, amsthm}  % math
\usepackage{xcolor} % colour
%\usepackage{empheq} % emphasize equations 
%\usepackage[most]{tcolorbox} % config box
\usepackage{hyperref} % clickable files
\usepackage{listings} % environment for code
\usepackage{setspace} % space between lines

% ==================
% 	document information
% ==================
%% agregar datos del alumno
\newcommand{\firststudent}{Nombre Apellido} % modificar
\newcommand{\secondstudent}{Nombre Apellido} % modificar % add sutdent information (name and lastname) and store in \primeralumno and \secondalumno
\newcommand{\course}{Control no lineal}
\newcommand{\labnumber}{1}
\newcommand{\labname}{Puntos de equilibrio}
\newcommand{\period}{2022 - 2}

% =============
% 	pdf information
% =============
\hypersetup{%
pdftitle   = {lab~\labnumber-\labname},
pdfauthor  = {\firststudent~and~\secondstudent},
colorlinks = true,
linkcolor  = blue,
urlcolor   = blue,
citecolor  = blue
}


% ==============
% page configuration
% ==============
\geometry{% 140 mm, 240 mm
	%showframe, % show frames, useful to make figures
  	papersize={210mm, 300mm},
  	left   = 30mm,
  	right  = 30mm,
  	top    = 30mm,
  	bottom = 30mm,
  	heightrounded, % ensures the height is adjusted so that an integer number of lines are accommodated in the text block.
}
\pagestyle{fancy}
\fancyhf{}
\setlength{\headheight}{15pt}
\lhead{\textit{\course}}
\rfoot{-~\thepage~-}
\lfoot{\textit{laboratorio~\labnumber:~\labname}}
\renewcommand{\headrulewidth}{0.5pt}
\renewcommand{\footrulewidth}{0.5pt}

%\renewcommand{\thesection}{\arabic{section}}
%\renewcommand{\thesubsection}{\thesection.\arabic{subsection}}
%\renewcommand{\thesubsubsection}{\thesubsection.\arabic{subsubsection}}
\setstretch{1.2}
\captionsetup{justification=centering} % center labels


% ===================
% 	make front page
% ===================
\newcommand{\makefrontpage}{%
	\thispagestyle{empty}
	\begin{center}
		\textbf{{\huge Universidad de Ingeniería y Tecnología}\\ [0.5cm]
			{\LARGE Departamento de Ingenier\'ia  Mecatr\'onica}\\[2cm]
		}
		{\includegraphics[width=.4\textwidth]{images/utec_logo}}\\[2cm]

		{\LARGE \textbf{\course}}\\[2em]
		{\Large \textbf{Laboratorio \labnumber}}\\[1em]
		{\Large \labname}\\[1.5cm] 		
		{\Large \textbf{Instructor de laboratorio:}}\\ \vspace{1em}
		{\Large Jhon~Charaja} \\[1cm]
		{\Large \textbf{Asistente de laboratorio:}} \\ \vspace{1em} 
		{\Large Ricardo~Terreros}\\[1cm]
		{\Large \textbf{Estudiantes:}} \\ \vspace{1em} 
		{\Large \firststudent \\ \vspace{.5em} \secondstudent}\\[1.5cm]
		{\Large \textbf{Lima - Perú}} \\[1.5em]
		{\LARGE \textbf{\period}}
	\end{center}
	\newpage
}

\renewcommand{\maketitle}{%
	\begin{center}
	\vspace*{1cm}
	{\huge \textbf{Laboratorio~\labnumber}} \\ \vspace{1em}
	{\LARGE \textbf{\labname}} \\ \vspace{1cm}
	\end{center}
}

% =========
% 	my colors
% =========
\definecolor{codegreen}{rgb}{0,0.6,0}
\definecolor{codegray}{rgb}{0.5,0.5,0.5}
\definecolor{codepurple}{rgb}{0.58,0,0.82}
\definecolor{backcolour}{rgb}{0.95,0.95,0.92}
