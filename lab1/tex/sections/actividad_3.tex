\subsection{Actividad 3: Diferencia entre modelo lineal y no lineal (16 pts)}
El modelo matemático del módulo con carga lineal  está dado por:
\begin{equation}
	\frac{J_\mathrm{eq} R_m}{k_t} \ddot{\theta} + k_m \dot{\theta} = u,
	\label{eq:act1_modulo_disco}
\end{equation}
donde $J_\mathrm{eq}$ es la inercia equivalente del sistema, $R_m$ es , $k_m$ es, $\theta$ es la posición angular del disco de masa lineal y $u$ es la señal de control. Con el objetivo de establecer el comportamiento temporal del sistema se define la siguiente ley de control
\begin{equation}
	u  = \frac{J_\mathrm{eq} R_m}{k_t} \ddot{\theta} (-2\lambda \dot{\theta} - \lambda^2 \theta) + k_m \dot{\theta}- \theta^2 - 50 \theta + 400,
	\label{eq:act1_ley_de_control}
\end{equation}
donde $\lambda$ es parámetro de control relacionado con el tiempo de establecimiento del sistema.

Desarrollar las preguntas de esta actividad con el archivo \texttt{simulink\_act\_1.xls}. Este archivo contiene tres bloques de subsistema: (i) \texttt{ley\_de\_control}, (ii) \texttt{modelo\_dinámico} y (iii) \texttt{guardar\_datos}. Por un lado, el bloque \texttt{ley\_de\_control} contiene los valores de cada parámetro necesario para calcular \eqref{eq:act1_ley_de_control}. Por otro lado, el bloque \texttt{modelo\_dinámico} contiene el bloque de escritura de Quanser para enviar la señal de control y el bloque de lectura para recibir la posición del motor . Finalmente, el bloque \texttt{guardar\_datos} guarda los estados del sistema y el tiempo de simulación. Para acceder al tiempo de simulación, posición y velocidad angular se debe usar \texttt{tiempo}, \texttt{x} y \texttt{dx} respectivamente.
\begin{enumerate}
	\item (2 pt) Obtener la ecuación del sistema en lazo cerrado.
	
	\item (2 pts) Determinar los puntos de equilibrio del del sistema en lazo cerrado.
	
	\item (2 pts) Implementar la ecuación del sistema en lazo cerrado. Usar el archivo \texttt{simulink\_act\_1.xls} y agregar imagen del contenido del bloque \texttt{ley\_de\_control}.
	
	\item (2 pts) Generar gráfica de posición respecto al tiempo del sistema en lazo cerrado para posición angular inicial de $20^{\circ}$, $30^{\circ}$, $50^{\circ}$ y $70^{\circ}$. Usar el archivo \texttt{simulink\_act\_1.xls} con las modificaciones del  bloque \texttt{ley\_de\_control} y usar la variable $\mathbf{x_0}= [\text{pos}, \text{ vel}]$ para establecer la condición inicial del sistema. 
	
	\item (1 pt) Indicar en que posición se establece el sistema en lazo cerrado para cada condición inicial.
	
	\item (1 pt)  Explicar porque el sistema en lazo cerrado se establece en esa posición.

	\item (2 pts) Generar gráfica de posición respecto al tiempo del sistema en lazo cerrado para posición angular inicial de $0^{\circ}$, $-20^{\circ}$, $-30^{\circ}$ y $-80^{\circ}$. Usar el archivo \texttt{simulink\_act\_1.xls} con las modificaciones  del bloque \texttt{ley\_de\_control} y usar la variable $\mathbf{x_0}= [\text{pos}, \text{ vel}]$ para establecer la condición inicial del sistema. 	
	
	\item (1 pt) Indicar en que posición se establece el sistema en lazo cerrado para cada condición inicial.
	
	\item (1 pt)  Explicar porque el sistema en lazo cerrado se establece en esa posición.	
	
	\item (2 pts) Explicar porque el sistema en lazo cerrado no se establece en el punto de equilibrio $x_e=10^{\circ}$ a pesar de empezar en posiciones iniciales cercanas al punto de equilibrio. 

\end{enumerate}
